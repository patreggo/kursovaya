\section*{ВВЕДЕНИЕ}
\addcontentsline{toc}{section}{ВВЕДЕНИЕ}

В современном цифровом обществе онлайн-торговля занимает важное место, предоставляя потребителям удобство и множество вариантов выбора. В этом контексте создание веб-приложения для онлайн магазина становится актуальной задачей, направленной на обеспечение эффективного и удобного взаимодействия между продавцами и покупателями. Данная курсовая работа посвящена разработке и анализу веб-приложения для онлайн магазина кроссовок. Основной целью исследования является создание функционального, эргономичного и привлекательного веб-приложения, способного эффективно удовлетворять потребности клиентов в приобретении обуви, а также управлять продуктовым ассортиментом и обработкой заказов.

В процессе выполнения курсовой работы будет проведен анализ требований к веб-приложению, выбраны соответствующие технологии разработки, спроектированы интерфейс и база данных, а также реализованы ключевые функциональные элементы, такие как каталог товаров, система заказов и управление акциями. 

Предполагается, что результаты исследования позволят не только успешно реализовать веб-приложение, но и провести оценку его эффективности и потенциала для улучшения пользовательского опыта в сфере онлайн покупок обуви.

Основной посыл исследования заключается в стремлении к созданию функционального, эстетически приятного и технологически инновационного веб-приложения, способного не только удовлетворять потребности клиентов, но и эффективно управлять продуктовым портфелем, организовывать процесс заказов и предоставлять дополнительные сервисы.

\emph{Цель настоящей работы} – разработка веб-приложения онлайн магазина, обеспечивающего покупателям удобный и интуитивно понятный доступ к широкому ассортименту кроссовок.\emph{следующие задачи:}
\begin{itemize}
\item провести анализ предметной области;
\item разработать концептуальную модель web-сайта;
\item спроектировать web-сайт;
\item реализовать сайт средствами web-технологий.
\end{itemize}

\emph{Структура и объем работы.} Отчет состоит из введения, 4 разделов основной части, заключения, списка использованных источников, 2 приложений. Текст выпускной квалификационной работы равен \formbytotal{page}{страниц}{е}{ам}{ам}.

\emph{Во введении} сформулирована цель работы, поставлены задачи разработки, описана структура работы, приведено краткое содержание каждого из разделов.

\emph{В первом разделе} на стадии описания технической характеристики предметной области приводится сбор информации о деятельности компании, для которой осуществляется разработка сайта.

\emph{Во втором разделе} на стадии технического задания приводятся требования к разрабатываемому сайту.

\emph{В третьем разделе} на стадии технического проектирования представлены проектные решения для web-сайта.

\emph{В четвертом разделе} приводится список классов и их методов, использованных при разработке сайта, производится тестирование разработанного сайта.

В заключении излагаются основные результаты работы, полученные в ходе разработки.

В приложении А представлен графический материал.
В приложении Б представлены фрагменты исходного кода. 
