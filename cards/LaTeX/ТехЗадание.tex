\documentclass[a4paper]{article}                                                       
\usepackage[utf8]{inputenc}
\usepackage[T2A]{fontenc}
\usepackage[russian,english]{babel}
\begin{document}
\section{Техническое задание} 
\subsection{Основание для разработки}

Основной задачей выпускной квалификационной работы является разработка сайта-блога.

\subsection{Цель и назначение разработки}

Задачами данной разработки являются:
\begin{itemize}
\item создание раздела сайта с постами
\item реализация формы для добавления постов.
\end{itemize}

\subsection{Требования пользователя к интерфейсу web-сайта}

Сайт должен включать в себя:
\begin{itemize}
    \item Навигацию по постам.;
    \item Авторизацию;
    \item Доступы для администратора, пользователя
\end{itemize}

Композиция шаблона сайта представлена на рисунке ~\ref{templ:image}.

\begin{figure}[ht]

\caption{Композиция шаблона сайта}
\label{templ:image}
\end{figure}
%\vspace{-\figureaboveskip} % двойной отступ не нужен (можно использовать, если раздел заканчивается картинкой)

\subsection{Моделирование вариантов использования}
На основании анализа предметной области в программе должны быть реализованы следующие прецеденты:
\begin{enumerate}
\item Просмотр постов
\item Комментирование постов
\item Удаление тем и постов.
\item Поиск по сайту.
\end{enumerate}

\subsection{Требования к оформлению документации}

Разработка программной документации и программного изделия должна производиться согласно ГОСТ 19.102-77 и ГОСТ 34.601-90. Единая система программной документации.
\end{document}