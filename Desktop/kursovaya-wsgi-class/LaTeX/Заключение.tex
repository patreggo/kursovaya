\section*{ЗАКЛЮЧЕНИЕ}
\addcontentsline{toc}{section}{ЗАКЛЮЧЕНИЕ}

Преимущества аддитивных технологий заключается в разнообразии процессов, позволяющих применять их в различных областях производства. Существенным ограничением же является и экономическая составляющая, которая не позволит внедрить аддитивное производство повсеместно.
  
Компании, видя, как развиваются информационные технологии, пытаются использовать их выгодно для своего бизнеса, запуская свой сайт для того, чтобы заявить о своем существовании, проинформировать потенциального клиента об услугах или продуктах, которые предоставляет. 
Для продвижения компании «Русатом – Аддитивные технологии» был разработан веб-сайт на основе системы «1С-Битрикс: Управление сайтом».

Основные результаты работы:

\begin{enumerate}
\item Проведен анализ предметной области. Выявлена необходимость использовать 1С-Битрикс.
\item Разработана концептуальная модель web-сайта. Разработана модель данных системы. Определены требования к системе.
\item Осуществлено проектирование web-сайта. Разработана архитектура серверной части. Разработан пользовательский интерфейс web-сайта.
\item Реализован и протестирован web-сайт. Проведено модульное и системное тестирование.
\end{enumerate}

Все требования, объявленные в техническом задании, были полностью реализованы, все задачи, поставленные в начале разработки проекта, были также решены.

Готовый рабочий проект представлен адаптивной версткой сайта. Сайт находится в публичном доступе, поскольку опубликован в сети Интернет.  
